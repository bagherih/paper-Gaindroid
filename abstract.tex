\begin{abstract}

\begin{comment}

Android has been the leading mobile operating system since
late 2010. Despite its dominance in the mobile market share,
it is common to find users talking about their poor
experience using apps in online communities due to run-time
crashes, which may lead them to use rival apps. This
incompatibility problem is due primarily to the frequent
changes in the Android framework, where every new release
adds new functionalities to improve performance and
security. Android API evolves rapidly, and apps often do not
offer proper support to either previous or most recent
Android API levels, which leads to run-time crashes.  In
this paper, we present \@approach, an automated approach
that verifies if an app may cause run-time crashes. We
compared \@approach against the state of the art tools and
verified that our approach identifies more API invocation
and API callback mismatches true positives with less false
positives. In addition, \@approach detects if the crash may
occur due to permission request or revocation mismatch. We
also conducted a large study over 3,571 real-world apps
downloaded from Androzoo and F-Droid to compare the
scalability of our approach to the state-of-the-art tools.
We find that \@approach is more effective than the
state-of-the-art techniques in detecting API
incompatibilities, it identifies a new set of errors related
to permissions and it is 4 times faster than current tools.

\end{comment}

\begin{comment}
Android has been the leading mobile operating system since late 2010. Despite
its dominance in the mobile market share, it is common to find users talking
about their poor experience using apps in online communities due to run-time
crashes, which may lead them to use rival apps. This incompatibility problem is
due primarily to the frequent changes in the Android framework, where every new
release adds new functionalities to improve performance and security. Android
API evolves rapidly, and apps often do not offer proper support to either
previous or most recent Android API levels, which leads to run-time crashes.
In this paper, we present \@approach, an automated approach that verifies if an
app is using Android APIs and dangerous permissions that may cause run-time
crashes. We conducted a study with 1,886 apps and compared our approach against
the state-of-the-art tools. We find that \@approach is more effective than the
state-of-the-art techniques in detecting general API incompatibilities and the
permission-induced compatibility issues that can lead to run-time crashes.
\end{comment}
\sloppy
With the ever-increasing popularity of mobile devices over the last decade,
mobile applications and the frameworks upon which they are built frequently
change, leading to a confusing jumble of devices and applications utilizing
differing features even within the same framework. For Android apps and
devices---the largest such framework and marketplace---mismatches between the
version of the app API installed on a device and the version targeted by the
developers of an app running on that device can lead to run-time crashes,
providing a poor user experience. In this paper, we present \@approach, a new
analysis approach that automatically detects three types of mismatches to which an
app may be vulnerable across versions of the Android API it declaratively
supports. We applied \@approach to 3,590 apps and compared the results of our
analysis against state-of-the-art tools. The experimental results corroborate
its ability to remarkably outperform the existing analysis techniques in terms
of both the number and type of mismatches correctly identified as well as
run-time performance of the analysis.

\end{abstract}
