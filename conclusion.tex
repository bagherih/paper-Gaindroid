%%\vspace{-0.3cm}
\section{Conclusion and Future Work}\label{sec-conclusion}
%%\vspace{-0.1cm}

This paper presents \@approach, a novel approach and
accompanying tool-suite for efficient analysis of various
types of crash-inducing Android compatibility issues.  The
experimental results comparing \@approach with the
state-of-the-art in Android incompatibility detection
corroborate its ability to efficiently detect more potential issues, yielding fewer false positives and
executing in a fraction of the time needed by the other
techniques.  Applying \@approach to thousands of real-world
apps from various repositories reveals that as many as 42\%
of the analyzed apps are prone to API invocation mismatch,
20\% can crash due to API callback mismatch, and 40\% of the
apps can suffer from crashes due to permissions-related
mismatch, indicating that such problems are still prolific
in contemporary, real-world Android apps. 
%and that
%\@approach can successfully perform large scale analysis.

As future work, we plan  to explore the trade-off between increasing analysis precision, e.g.,
through incorporating additional information such as CCFG for the compatibility analysis, and higher analysis overhead. In addition, we plan to develop a complementing code synthesizer to help repair apps that do not properly handle detected mismatches. 

\begin{comment}
\commentcs{
We also plan to develop new detectors to identify other events that can lead to dependability and security issues. 
Another idea
is to provide guidance to users to replace the use of
possibly outdated or deprecated APIs with more updated
ones.  For example, our system can recommend a
developer to replace the old SSL API with the new one.
While the old one still works,  it is less secure than
the newer API.
}
\end{comment}
 

 
%\@approach will be made available to other researchers and practitioners upon the publication of this work.
