\section{Related Work}\label{sec-related}
%Android incompatibility issues have received a lot of
%attention recently. Here, we provide a discussion of the
%related efforts in light of our research.

\textbf{API evolution}. A large body of existing
research focuses on the evolving nature of APIs, which
is an important aspect of software
maintenance~\cite{mcdonnell2013empirical},
\cite{bavota2015impact}, \cite{ACRYL_Scalabrino2019},
\cite{li2018characterising},
\cite{lamothe2018exploring}, \cite{linares2013api},
\cite{Luo2018},
\cite{Fazzini:2017:ACI:3155562.3155604},
\cite{mahmoudi2018android}, \cite{mutchler2016target}.
These research efforts explore problems due to API
changes. Among others, McDonnell et
al.~\cite{mcdonnell2013empirical} studied Android's
fast API evolution (115 API updates/month), and noticed
developers' hesitation in embracing the fast-evolving
APIs because they can be more defect-prone than other
types of changes, which might cause application
instability and potential
vulnerabilities~\cite{linares2013api}.  
Bavota et al.  \cite{bavota2015impact}
showed that applications with higher user ratings use
APIs that are less change- and fault-prone compared to
the applications with lower ratings. 
%Mutchler et al.~\cite{mutchler2016target} explored the consequences
%of running applications targeted to older Android
%versions on devices with recent Android versions,
%and how it can introduce serious security issues. 
Li et al.  \cite{li2018characterising} investigated the
frequency with which deprecated APIs are used in the
development of Android apps, considering the deprecated
APIs' annotations, documentations, and removal
consequences along with developers' reactions to APIs
deprecations. 
%
%
%
%\textcolor{red}{
%The results of these studies, identifying potential issues
%caused by API evolutions, are the primary motivations for
%this work.
%
These prior efforts clearly motivate the need to address
issues relating to API evolution.  However, their
approaches do not provide detailed technical solutions or
methods to systematically detect the root causes of these
problems.  \textsc{\@approach}, on the other hand, is
designed to be effective at detecting crash-leading API
related issues.

%}

\begin{comment}

\textbf{Android testing}. The other line of related research
efforts involves techniques that have been proposed to
expose potential runtime errors, mainly using various
testing methods ~\cite{moran2016automatically},
\cite{khan2018crashsafe}, \cite{moran2017crashscope},
\cite{gomez2013reran}, \cite{amalfitano2012using},
\cite{amalfitano2015mobiguitar}. For example, Moran et al.
\cite{moran2017crashscope} presented a systematic input
generation driven by both static and dynamic analyses to
trigger app crashes. 
%Automatically discovering, reporting and reproducing
%android application crashes
Given such automatically generated inputs, it produces a
crash report containing screenshots, detailed crash
reproduction steps, the captured exception stack trace, and
a fully replayable script that automatically reproduces the
crash on a target device. Amalfitano et al.
\cite{amalfitano2015mobiguitar} extracted relationships
between GUI widgets to automatically generate test cases,
with the goal of yielding possible runtime errors raised by
the automatic generated test cases.
\textcolor{pblue}{However, testing by itself cannot uncover
the underlying reasons for application crashes. These
techniques generate inputs for the target app to exercise
illegal sequences of events that may lead to a crash. In
these works, detection of incompatible Android APIs across
different levels of the framework is missing. \@approach
however, addresses and detects this problem in Android
framework, which exists due to its fast paced API
evolution.}



\textbf{Android fragmentation}.  The other relevant thrust
of research has focused on investigating the Android
ecosystem by running different custom Android distributions
on different hardware to identify potential application
instability and uncovering the causes
\cite{han2012understanding}, \cite{li2015characterizing},
\cite{pathak2011bootstrapping},
\cite{liu2014characterizing}, \cite{zhou2014peril},
\cite{wei2016taming}, \cite{aafer2016harvesting}.  Aafer et
al. \cite{aafer2016harvesting} investigated how modifying
the operating system can introduce security problems within
the mobile OS. Han et al. \cite{han2012understanding}
studied the bug reports related to HTC and Motorola devices
in the Android issue tracking system, and discovered that
Android ecosystem was fragmented, meaning that applications
might behave differently when installed on phones from
different vendors. Liu et al. \cite{liu2014characterizing}
observed that a noticeable percentage of Android performance
bugs occur only on specific devices and platforms.  Moran et
al. \cite{moran2017crashscope} presented a systematic input
generation driven by both static and dynamic analyses to
trigger app crashes. 
%Automatically discovering, reporting and reproducing
%android application crashes
Given such automatically generated inputs, it produces a
crash report containing screenshots, detailed crash
reproduction steps, the captured exception stack trace, and
a fully replayable script that automatically reproduces the
crash on a target device.  
%
%\textcolor{red}{

These research efforts primarily focused on
behavioral differences when an app is installed on different
operating systems and/or hardware platforms.  They mainly
rely on hardware specifications and changes in the Android
documentation to uncover potential compatibility or
behavioral issues. Therefore, these approaches are not
useful when such platform related information is incomplete,
inconsistent, or unavailable.  Furthermore,  applying these
approaches to test an application on the entire vast
hardware ecosystem of Android devices may not be feasible
due to exponentially large system configurations.  Our work,
on the other hand, focuses on a more tractable and important
problem due to API evolution and how it can affect the apps
and their performance regardless of the operating system
distribution or the hardware the applications are running
on.

\end{comment}

\textbf{API incompatibility}. 
In Table~\ref{tab:table_comparison_tools}, we compare the
detection capabilities of \textsc{\@approach} against the
current state-of-the-art approaches. It is important to
stress that \@approach is the only solution that 
can automatically detect API invocation
compatibility issues (API), API callback compatibility
issues (APC), and permission-induced compatibility issues
(PRM). %We continue to discuss some of the closely related works to ours and the differences between those works and ours.
%compared to the one issue identified by each of the other
%approaches.

\begin{table}%{R}{0.40\textwidth}
    %\begin{table}
\centering
\caption {Comparing \@approach to the state-of-the-art of compatibility detection techniques.} %, \@approach is the only solution that provides  
%\vspace{-0.2cm}
%\scriptsize{
\footnotesize{
%\small{
\begin{tabular}{l|c|c|c} 
    \hline
    \hline    
    \rule{0pt}{3ex}
    & API & APC & PRM     \\ 
    \hline \hline
   \rule{-3pt}{3ex}
    \textsc{CiD}~\cite{lili2018cid}              & $\cmark$ & $\xmark$ & $\xmark$\\
    \textsc{Cider}~\cite{huang2018understanding} & $\xmark$ & $\cmark$ & $\xmark$\\
    IctApiFinder~\cite{he2018understanding} 		& $\cmark$ & $\xmark$ & $\xmark$\\    
    \textsc{Lint}~\cite{linttips}                & $\cmark$ & $\xmark$ & $\xmark$\\
    \textsc{\@approach}                 & $\cmark$ & $\cmark$ & $\cmark$ \\
    \hline
    \hline
\end{tabular}
%\vspace{-0.2cm}
}

\label{tab:table_comparison_tools}
%\end{table}

\end{table}

Wu et al.~\cite{wu2017measuring} investigated side
effects that may cause runtime crashes even within an
app's supported API ranges, inspiring subsequent work.
%
Huang et al.~\cite{huang2018understanding} aimed to
understand callback evolution and developed
\textsc{Cider}, a tool capable of identifying API
callback compatibility issues. However,
\textsc{Cider}'s analysis relies on manually built
PI-GRAPHS, which are models of common compatibility
callbacks of four classes: Activity, Fragment, Service
and WebView. \textsc{Cider} thus does not deal with
APIs that are not related to these classes or
permission induced mismatches. 
%As this tool focuses on these predefined callback
%classes, Moreover, by only focusing on callback
%classes, 
As such, their reported result is a subset of ours. In
addition, \textsc{Cider}'s API analysis is based on the
Android documentation, which is known to be
incomplete~\cite{wu2017measuring}. Our work, on the
other hand, automatically analyzes each API level in
its entirety to identify all existing APIs. This allows
our approach to be more accurate in detecting actual
changes in API levels, as there are frequent platform
updates and bug fixes. As a result, and as confirmed by
the evaluation results, our approach features much
higher precision and recall in detecting compatibility
issues.
%\textsc{Cider} also relies on constructing Callback
%Control Flow Graph, while our analysis simply works on
%an existing method call graph and data flow graphs
%generated by \textsc{Jitana} to help us uncover
%sources of potential issues.}  % but it comes at a
%cost of higher manual efforts and possibly longer
%analysis time. However, while being able to complete
%each analysis in just a few seconds.  Another
%difference between the two approaches is that their
%analysis is based on \textsc{Soot}, which requires the
%Dex code be converted to Jimple.

Lint \cite{linttips} is a static analysis tool
introduced in ADT (Android Development Tools) version
16. One of the benefits of Lint is that the plugin is
integrated with the Android Studio IDE, which is the
default editor for Android development. The tool checks
the source code to identify potential bugs such as
layout performance issues, and accessing API calls that
are not supported by the target API version. However,
the tool generates false positives when verifying
unsupported API calls (e.g., when an API call happens
within a function triggered by a conditional
statement). Another disadvantage is that it requires
the availability of the original Java source code, and
it does not analyze APKs.  In addition, \textsc{Lint}
requires the project to be first built in the Android
Studio IDE before conducting the analysis.  Unlike
\textsc{Lint}, \@approach operates directly on Dex
code. While \textsc{Lint} claims to be able to detect
API incompatibility issues, our experimental results as
well as the results obtained by Huang et al.  indicate
that \textsc{Lint} is not as effective as \@approach. %or \textsc{Cider}.

Li et al.~\cite{lili2018cid} provided an overview of the
Android API evolution to identify cases where compatibility
issues may arise in Android apps. They also presented
\textsc{CiD}, an approach for identifying compatibility
issues for Android apps. This tool models the API lifecycle,
uses static analysis to detect APIs within the app's code,
and then extracts API methods from the Android framework to
detect backward incompatibilities.  \textsc{CiD} supports
compatibility analysis upto the API level 25.  In
comparison, \@approach offers automated extraction of the
API database, and thereby supports up to the most recent
Android platform (API level 29).  Moreover, in contrast to
\@approach, \textsc{CiD} did not consider incompatibilities
regarding the runtime permission system.

Wei et al.~\cite{wei2016taming} conducted a study to
characterize the symptoms and root causes of compatibility
problems, concluding that the API evolution and problematic
hardware implementations are major causes of compatibility
issues. They also propose a static analysis tool to detect
issues when invoking Android APIs on different devices.
Their tool, however, needs manual work to build API/context
pairs, of which they only define 25. Similar to our prior
discussion of work by Huang et al., the major difference
between our work and this work is that our approach can
focus on all API methods that exist in an API level. Again,
the result reported by their approach would be a subset of
our detected issues.
