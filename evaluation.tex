\def \apitotalALL {68,268}
\def \apctotalALL {2,115}

\def \apicountALL {1,471}
\def \apiprcntALL {41.19\%}
\def \apicountZOO {878}
\def \apicountFDR {593}

\def \apccountALL {761}
\def \apcprcntALL {20.05\%}
\def \apccountZOO {464}
\def \apccountFDR {252}

\def \andzooct {2,180}
\def \fdroidct {1,391}

\def \prqcount {224}
\def \prqprcnt {12.34\%}
\def \prqtotal {1,815}

\def \prvcount {1,206}
\def \prvprcnt {68.68\%}
\def \prvtotal {1,756}

\def \prmcount {1,430}
\def \prmprcnt {40.04\%}

\def \rqoneapps {19}

\section{Experimental Evaluation}
\label{sec-eval}

This section presents the experimental evaluation of~\@approach. 
%\textcolor{blue}{
We have implemented \@approach's static analysis capability on top of the \textsc{Jitana} framework~\cite{tsutano2017efficient}, %\textsc{Jitana} is a high-performance hybrid analysis tool for Android. It works directly on Dalvik executable (dex) files contained in each APK.
which is hybrid analysis tool for Android.
We also used \textsc{APKTool} \cite{apktool} for %reverse engineering Android APK files and 
extracting apps' manifest files.
As a result, our approach implementation only requires the availability of Android executable files, and not the original source code. \@approach, thus, can be used not only by developers, but also by end-users as well as third-party reviewers to assess the compatibility of their mobile apps.
%, which enables our tool to work on non-open source applications whose APKs can be found online. We also need to decompile the APK file to extract the manifest file, and to that end we use \textsc{APKTool} \cite{apktool}. 
%We further modified \textsc{Jitana} to decode dex files using Android version 6.0.0, which is the version in which the new runtime permissions system is introduced. We also extended \textsc{Jitana} to perform inter-procedural dataflow analysis, which enabled us to detect more API related issues within different methods of an Android app.
%}
\@approach's tool and the experimental data are available at the project website~\cite{GainDroid}.

We used the \@approach apparatus for carrying out the experiments. In our evaluation, we address the following research questions:

%Our evaluation of \@approach addresses the following research questions:

\textbf{RQ1.} \textit{Accuracy:} What is the overall accuracy of \@approach in detecting compatibility issues compared to the other state-of-the-art techniques?

\textbf{RQ2.} \textit{Applicability:} How well does \@approach perform in practice? Can it find compatibility issues in real-world applications?

\textbf{RQ3.} \textit{Performance:} What is the performance of \@approach's analysis to identify sources of compatibility issues?

\subsection{Objects of Analysis}

Our experimental objects are a set of Android apps
drawn from different sources.  

To evaluate the accuracy
of our analysis technique and compare it against the
other compatibility analysis tools, we used two suites
of benchmark apps,
\textsc{CiD}-Bench~\cite{lili2018cid} and
\textsc{Cider}-Bench~\cite{huang2018understanding},
developed independently by other research groups.
%the DroidBench [4] and ICC-Bench [8] suites of
%benchmarks, two sets of Android applications
%containing ICC based privacy leaks for which all
%vulnerabilities are known in advance—establishing a
%ground truth.
\textsc{CiD}-Bench contains seven benchmark apps and
\textsc{Cider}-Bench contains 20 apps, for which all
compatibility issues are known in advance--establishing
a ground truth. 
\commentty{The ICSE reviewer 1 noted that Cider did not evaluate
recall, so he/she does not think \textsc{Cider}-Bench
has ground truths. We should clarify.}
%We obtained seven benchmark apps from Li et
%al.~\cite{lili2018cid} and 20 apps from Huang et
%al.~\cite{huang2018understanding} as our objects of
%analysis, for which all compatibility issues are known
%in advance--establishing a ground truth.
The collection includes apps of varying sizes ranging
from 10,400 to 294,400 lines of Dex code and up to tens
of thousands of methods. The benchmark apps both
support and target a variety of API levels, with
minimum levels ranging from 10 to 21 and targets
ranging from level 23 to 27.  
One of our baseline system, i.e. \textsc{Lint},
requires building the apps to perform the compatibility
analysis. Out of the 27 benchmark apps, eight apps
cannot be built\footnote{The benchmark apps were built using Gradle~\cite{Gradle}, which dropped support of some Android SDK tool chains. Even with the appropriate SDKs in place on two different systems, Gradle were unable to build the apps.}; therefore, they are excluded from the
analysis, leaving the total of \rqoneapps~apps used in
our~comparative~study.
Using the same benchmark
apps as prior research allows us to compare our results
against them and help eradicate internal threats to the
validity of our results.

To evaluate the implications of our tool in practice, we collected over 3,000 real-world Android apps from the following two sources:
%To further evaluate the applicability of our tool in practice, we collected a set of real-world Android apps from two repositories of FDroid~\cite{fdroid} and AndroZoo~\cite{allix2016androzoo}. 
(1) FDroid~\cite{fdroid} is a software
repository that contains free and open source Android
apps.  Our collection of subject systems includes all
\fdroidct\ apps available from the FDroid repository. 
(2) We also include 2,300 apps from AndroZoo~\cite{allix2016androzoo}, a growing repository of Android apps
collected from various sources, including the official Google Play store. We were
unable to build 120 of the apps from AndroZoo so we
excluded them from our analysis, leaving \apptotal~apps
in total.

\subsection{Variables and Measures}

\subB{Independent Variables.}
Our independent variables involve baseline techniques used
in our study to perform the analysis of compatibility issues.
These techniques include {\sc CiD}~\cite{lili2018cid}, {\sc
Cider}~\cite{huang2018understanding}, and
\textsc{Lint}~\cite{linttips}.

\textsc{CiD} represents a state-of-the-art in detecting
Android compatibility issues. It has been publicly released,
and we are able to obtain the tool and compile it in our
experimental environment.  We use it as the baseline system
to answer RQ1 and RQ3.

{\sc Cider} is another state-of-the-art approach developed to analyze API compatibility issues. Unfortunately, it is not available in  either source or binary forms at the
time of writing this article.  As such, we rely on their results as reported in \cite{huang2018understanding} to answer RQ1 and RQ3.

\textsc{Lint} is a static analysis technique, shipped
with the Android Development Tools (ADT), to examine
code bases for potential bugs, including incompatible
API usages.  \textsc{Lint} performs the compatibility
analysis as part of building apps, and thus requires the app source code to conduct the analysis. We use \textsc{Lint} to answer RQ1 and RQ3.

We also considered {\sc
IctApiFinder}~\cite{he2018understanding} as a possible
baseline technique. {\sc IctApiFinder} was introduced
at about the same time as \textsc{Cider}.
Unfortunately, the tool is not publicly available and
our attempts to contact the authors to request access
were unsuccessful. Therefore, we did not use it in our
study.

\subB{Dependent Variables.} As dependent variables, we chose
metrics allowing us to answer each of our three research
questions.

To measure accuracy, we compare the number of detected compatibility issues with known issues as reported by prior work~\cite{huang2018understanding,lili2018cid}. For each analysis technique, we report true and false positives and false
negatives thereof in detecting compatibility issues of the apps under analysis. Lastly, we report precision, recall and F-measure for each technique.


%We then report true (\tp) and false (\fp) positives and false negatives (\fn) based on comparisons to results from baseline techniques. Lastly, we report precision, recall and F-measure.


\newcommand{\timelbl}{Secs.}
%\begin{sidewaystable}
%\scriptsize
%\centering
%\begin{threeparttable}[c]
\begin{table}%[b!]
%\begin{tiny}
\caption {\label{tab:tab-results} Comparison between \@approach, \textsc{CiD}, \textsc{Cider}, and \textsc{Lint}. TP, FP and FN are~represented~by~symbols \tp, \fp, \fn, respectively. X(\#) indicates the number \# of detected instances for the~corresponding~symbol~X.}
%\vspace{-0.2cm}
%\small{
\scriptsize{
\begin{tabular}{l|ll|ll|ll} 
\hline
\hline
\rule{0pt}{3ex}
     & \multicolumn{2}{|c}{\sc \footnotesize{\@approach}} & \multicolumn{2}{|c}{\sc \footnotesize{CiD+Cider}} & \multicolumn{2}{|c}{\sc \footnotesize{Lint}} \\
 \multicolumn{1}{c|}{{\footnotesize App}} 
 & \multicolumn{1}{l}{API} & \multicolumn{1}{l|}{APC} 
 & \multicolumn{1}{l}{API} & \multicolumn{1}{l|}{APC} 
 & \multicolumn{1}{l}{API} & \multicolumn{1}{l}{APC} \\
\hline
\hline
\rule{0pt}{3ex}
 AFWall+         &  (\tp9)          & (\tp7)      & (\fn9)         & \tp(\fn6) &  \tp(\fn8) & (\fn7)          \\ 
 \midrule
 DuckDuckGo      &                  & \fn       & (\fp3)           & \tp         &               & \fn              \\ 
 \midrule
 \multirow{2}{*}{FOSS Browser}    &                  & \multirow{2}{*}{(\tp7)}   & \multirow{2}{*}{(\fp4)}       & \multirow{2}{*}{(\fn7)}     &  & (\fp3)  \\ 
    &                  &    &        &      &  & (\fn7)  \\ 
    \midrule
 \multirow{2}{*}{Kolab notes}     &  (\tp3)    &          & (\tp3)  & \multirow{2}{*}{\fp}        & \multirow{2}{*}{(\fn3)}       &              \\ 
      &  (\fp9)    &          & (\fp13)  &        &        &              \\ 
      \midrule
 \multirow{2}{*}{MaterialFBook}   &  (\tp11)&              & (\tp14)   &             & \multirow{2}{*}{(\fn14)}      &               \\
    &  \fp(\fn3)&              & (\fp17)   &             &       &               \\ 
    \midrule
 NetworkMonitor  &                  & (\tp5)  &                      & (\fn5)    &                 & (\fn5)            \\ 
 \midrule
 NyaaPantsu      &                  & (\tp12) &                       & (\fn12)   &   & (\fn12)          \\ 
 \midrule
 \multirow{2}{*}{Padland}         &                  & \multirow{2}{*}{\fn}     &      \multirow{2}{*}{(\fp4)}           & \multirow{2}{*}{\tp}       &  & (\fp2)         \\ 
          &                  &      &                 &        &  & \fn         \\ 
          \midrule
 PassAndroid     &  (\tp9)          & (\tp3)  &      (\fn9)          & (\fn3)      & (\fn9)       & (\fn3)            \\ 
 \midrule
 \multirow{2}{*}{SimpleSolitaire} &  \multirow{2}{*}{\tp\fp}          & \multirow{2}{*}{(\tp2)}  &      \tp       & \multirow{2}{*}{\tp\fn}      & \multirow{2}{*}{\fn}          & (\fp2)     \\
  &          &   &      (\fp10)       &       &          & (\fn2)     \\ 
  \midrule
 SurvivalManual  &                  &         &     (\fp19)        &           &  &                   \\ \midrule
 Uber ride       &                  & (\tp4)  &      (\fp2)         & (\tp4)     & & \fp(\fn4)    \\
\hline
\hline
 Basic       & \tp    & &  \tp        &  & \fn    &   \\ \midrule
 Forward     & \tp    &  & \tp        &  & \fn    &   \\ \midrule
 GenericType & \tp        & & \tp           &  & \fn       &   \\ \midrule
 Inheritance & (\tp2) &  & (\tp2)   &    & (\fn2) &   \\ \midrule
 Protection  &     &  &       &  &       &   \\ \midrule
 Protection2 &     &  &       &   & \fn    &   \\ \midrule
 Varargs     & (\tp2) &  & (\tp2)  &    & (\fn2) &   \\
\hline
\hline
\rule{0pt}{3ex}\textbf{Precision:} & \multicolumn{1}{l}{\bf 79\%} & \multicolumn{1}{l|}{\bf 100\%} &  \multicolumn{1}{l}{\bf  27\%} & 
 \multicolumn{1}{l|}{\bf 89\%} & \multicolumn{1}{l}{\bf 100\%} & \multicolumn{1}{l}{\bf 0\%} \\ 
{\bf Recall:}                   & \multicolumn{1}{l}{\bf 93\%} & \multicolumn{1}{l|}{\bf  95\%} &  \multicolumn{1}{l}{\bf 59\%} & 
\multicolumn{1}{l|}{\bf 19\%} &  \multicolumn{1}{l}{\bf 2\%} & \multicolumn{1}{l}{\bf 0\%}  \\ 
{\bf F-Measure:}                & \multicolumn{1}{l}{\bf 85\%} & \multicolumn{1}{l|}{\bf  98\%} & \multicolumn{1}{l}{\bf  42\%} & 
\multicolumn{1}{l|}{\bf 31\%} & \multicolumn{1}{l}{\bf 4\%} & \multicolumn{1}{l}{\bf 0\%}  \\ 
\hline
\hline
%\rule{0em}{3ex}
%\parbox[t]{5mm}{\multirow{4}{*}{\rotatebox[origin=c]{90}{AZ/FD\tnote{3}}}} & \multicolumn{13}{l}{\rlap{API}\phantom{PRM}$\rightarrow$\ \apitotalALL\ mismatches in \apicountALL\ apps (\apiprcntALL)} \\
%& \multicolumn{13}{l}{\rlap{APC}\phantom{PRM}$\rightarrow$\ \apctotalALL\ mismatches in \apccountALL\ apps (\apcprcntALL)} \\
%& \multicolumn{13}{l}{PRM$\rightarrow$\ \prmcount\ apps exhibiting mismatch (\prmprcnt)} \\
%& \multicolumn{13}{l}{\rlap{Time}\phantom{PRM}$\rightarrow$\ 6.6s / app on average (1.5s min, 16.8s max)} \\
%\hline
%\hline

\end{tabular}
% GD(API-I): pre=82.69%, rec=93.48%, f-measure=87.75%
% GC(API-C): pre=100%, rec=95.24%, f-measure=97.56%
% CiD: pre=21.11%, rec=95%, f-measure=42.85%
% CIDER: pre=88.89%, rec=19.05%, f-measure=31.38%
%\begin{tablenotes}
%\item[1]{\sc CiD} does not detect API callback mismatches.
%\item[2]{\sc Cider} does not detect API invocation or permissions-related mismatches and was not timed for RQ3.
%\item[3]{\apptotal\ apps from Androzoo and FDroid, analyzed with \@approach only.}
%\end{tablenotes}
%\end{threeparttable}
%\end{tiny}
}
%\vspace{-0.5cm}
\end{table}
%\end{sidewaystable}




To measure applicability, we report the number of detected
compatibility issues in real-world apps. Finally, to measure
performance, we report the analysis time and the amount of memory used by each of the analysis techniques, i.e., \@approach, {\sc CiD}, and \textsc{Lint}.


\subsection{Study Operation}

We conduct our experiments on a MacBook Pro running
High Sierra 10.13.3 with an Intel Core i5 2.5 GHz CPU
processor and 8 GB of main memory. To answer RQ1 and
RQ2, we run each analysis once since the techniques are
based on static analysis.  To handle uncontrollable
factors in our experiments addressing RQ3
(performance), we repeat the experiments three times
and measure the amount of time required to perform the
analysis of each app using the analysis techniques,
each averaged over three attempts. Further, since
\textsc{Lint} needs to build the app before it can
perform the analysis, we perform four consecutive
analysis attempts with {\sc Lint}, and report the
average analysis time of the last three analyses. 

%To answer RQ3, we measure the amount of time required to perform the analysis of each app using \@approach and {\sc CiD}, each averaged over three attempts.  \textsc{Lint} needs to build the app before it can perform the analysis, so we performed four consecutive analysis attempts with {\sc Lint}, and report the average analysis time of the last three analyses.

 
%\begin{comment}
\subsection{Threats to Validity}

The primary threat to external validity in this study
involves the object programs utilized. In this work,
we have studied a smaller set of benchmark programs
developed and released by prior research
work~\cite{lili2018cid,huang2018understanding} so that
we can directly compare our results with their
previously reported results.  However, we also extend
our evaluation to employ over 3,590 complex real-world apps from
other repositories, which in turn enabled us to assess our system in
real-world scenarios, representative of those that
engineers and analysts are facing.

The primary threat to internal validity involves
potential errors in the implementations of \@approach
and the infrastructure used to run \textsc{CiD} and
\@approach.  To limit these, we extensively validated
all of our tool components and scripts to ensure
correctness.  By using the same objects as our baseline
systems we can also compare the results produced by our
approach with those previously reported to help with
ensuring correctness.

The primary threat to construct validity relates to the
fact that we study efficiency measures relative to
applications of \@approach, but do not yet assess
whether the approach helps software engineers or
analysts addresses dependability and security concerns
more quickly than current approaches.
%\end{comment}
