\section{Introduction}\label{sec-intro}
\sloppy
Android is the leading mobile operating system representing over 80\% of the
market share~\cite{androidmarketstatistics}. The meteoric rise of Android is
largely due to its vibrant app market~\cite{googleplayapps}, which currently
provisions nearly three million apps, with thousands added or updated on a
daily basis.  Android apps are developed using an application development
framework (ADF) that ensures apps devised by a wide variety of suppliers can
interoperate and coexist as long as they comply with the rules and
constraints imposed by the framework.  An ADF exposes well-defined application
programming interfaces (APIs) that manifest the set of extension points for
building the application-specific logic, setting it apart from traditional
software systems often realized as a monolithic, independent piece of code.

The Android ADF frequently evolves, with hundreds of releases from multiple
device vendors since 2010~\cite{AndroidReleases}.
Such rapid evolution leads to
incompatibilities in Android apps targeted to older
versions of the framework.  As a result, defects and vulnerabilities, especially
following ADF updates, continue to plague the dependability and security of
Android devices and apps~\cite{linares2014api,Update4}. A recent
study shows that 23\% of Android apps behave differently after a framework
update, and around 50\% of the Android updates have caused instability in previously working
apps and systems~\cite{Helppi2014}. This has been referred to as ``death on
update''~\cite{death1,death2,death3,death4,iOS,youtube}.
 As part of Android ADF version 6.0 (API-level 23), Google introduced a dynamic permission system.  
%Google introduced a dynamic permission system as part of Android ADF version 6.0 (API-level 23).  
Previously, the permission system was entirely static, with the user
granting all requested dangerous permissions at install
time.  The new permission system instead allows users to
grant or revoke permissions at
run-time~\cite{permissiongroups}.  This new system creates a
new class of permissions-related incompatibility~issues.

Recent research efforts have studied compatibility
issues~\cite{huang2018understanding,wei2016taming,wu2017measuring},
but existing detection techniques target only certain
types of APIs.  For example, Huang et
al.~\cite{huang2018understanding} only targets API
callback related to lifecycles; identifying them requires
significant manual labor~\cite{huang2018understanding}
and thorough inspection of incomplete
documentation~\cite{wu2017measuring}.  Furthermore,
none of the state-of-the-art techniques consider
incompatibilities due to the dynamic permission system.
The state-of-the-art compatibility detection techniques
also suffer from acknowledged frequent ``false alarms''
because of the coarse granularity at which they capture
API information.  The lack of proper support for
detecting compatibility issues can increase the time
needed to address such issues, often longer than six 
months~\cite{siliconangle}.
Finally, these techniques~\cite{huang2018understanding,he2018understanding} have been shown facing difficulties in handling large scale libraries, due to direct loading of the entire code base for analysis purposes.

%\commentty{what about scalability issues faced by existing work since high scalability is one of the contributions of GainDroid?}



%\sloppy In this paper, we present a \textbf{g}eneral, \textbf{a}utomated
\sloppy In this paper, we present a \textbf{s}calable, \textbf{a}utomated
\textbf{i}ncompatibility \textbf{n}o\textbf{t}ifier for Android, dubbed \@approach, which automatically 
%\sloppy In this paper, we present \@approach (\textbf{G}eneral, fully \textbf{A}utomated \textbf{I}ncompatibility \textbf{N}otifier for And\textbf{DROID}), which automatically %detects compatibility issues regarding the use of all Android APIs and the new runtime permissions system. \commentty{through static analysis?}
detects all types of API- and permission-induced mismatches.
Existing state-of-the-art compatibility detection techniques
require to either analyze the entire ADF codebase or manually model common compatibility callbacks of ADF classes, prior to detecting incompatibilities~\cite{huang2018understanding,lili2018cid,he2018understanding};
as such they face serious scalability issues that limit their abilities to detect complex types of incompatibilities. 
Different from all these techniques, \@approach overcomes such scalability issues by gradually loading and analyzing classes, wherein a reachability analysis is leveraged to stumble on all pertinent classes.


%Different from the state-of-the-art compatibility detection techniques that rely on the traditional compiler-based approaches for static analysis~\cite{huang2018understanding,lili2018cid}, \@approach takes a class loader-based analysis approach that gradually loads and analyzes classes, wherein a reachability analysis is leveraged to stumble on all pertinent classes. \@approach then analyzes the identified API call sites and their concomitant conditions to effectively detect compatibility issues regarding the use of all Android APIs and the new runtime permissions system.



%\@approach has several advantages over existing work. First, unlike prior techniques that focus on specific types of APIs, our approach has the potential to greatly increase the scope of analysis by automatically and effectively analyzing all of the APIs in an ADF version. Second, \@approach also analyzes platform code to uncover compatibility issues that exist within the framework; these issues cannot be detected by analyzing only the application code---an approach commonly taken by the state-of-the-art incompatibility detectors~\cite{huang2018understanding,linttips, lili2018cid}.
%\commentty{We need to revise the above sentence ---  ICSE reviewer 1 says CiD does not analyze only the application code.}
%Third, incrementally loading and analyzing classes allows our technique to be remarkably faster and more scalable than the state-of-the-art in compatibility detection.

\textcolor{blue}{
\@approach has several advantages over existing work. First, unlike prior techniques that focus on specific types of APIs, our approach has the potential to greatly increase the scope of analysis by automatically and effectively analyzing all of the APIs in an ADF version. 
Second, incrementally loading and analyzing classes allows our technique to be remarkably faster and more scalable than the state-of-the-art in compatibility detection.
Third, \@approach holistically analyzes application and ADF in tandem by gradually loading and analyzing classes as needed during the compatibility analysis. \@approach analysis, thus, can seamlessly move between the application code and the ADF code during the compatibility analysis. Whereas prior techniques first analyze the ADF code, the results of which are subsequently used to resolve the API usages~\cite{huang2018understanding,linttips, lili2018cid}.
}

Our evaluation of \@approach against the state-of-the-art analysis
techniques using thousands of real-world Android apps indicates that 
%Our experiments indicate that 
\@approach is up to 76\% more successful in detecting compatibility issues, while issuing significantly less (11-52\%) false alarms. 
It also successfully detects permission-induced mismatches that cannot be even detected by state-of-the-art techniques.
In addition, \@approach is  up to 8.3 times (four times on average) faster than
the state-of-the art techniques. 


%\commentty{Can we put the actual numbers here? E.g., the improvement of incompatibility detection,  performance gain, etc.}
%in the context of real-world Android
%apps collected from variety of repositories have been very
%The results, among other things, corroborate its
%ability to outperform existing analysis techniques both by
%remarkably decreasing the number of false positives and by
%detecting new classes of incompatibilities.  
To summarize,
this paper makes the following contributions:
 
 
\begin{itemize}
%\vspace{-0.2cm}
%{\color{green} 
\item \textit{General API and permission-induced incompatibility detection algorithms}: We
introduce novel algorithms that automatically detect all types of API incompatibilities and misuses of runtime permission
APIs to which an app may be vulnerable across ADF~versions.


\item \textit{Scalable incompatibility detection approach}:
We introduce a scalable analysis approach that can incrementally
load and analyze classes to handle large scale libraries in 
detecting incompatibility issues. 
%}

\item \textit{Publicly available tool implementation}: We develop a fully~automated
technology, \@approach, that effectively realizes~our
compatibility detection approach. We make \@approach publicly available to the research and education~community~\cite{GainDroid}.



\item \textit{Experiments}: We present results from
experiments run on 3,590 real-world apps and benchmark
apps, corroborating \@approach's ability in (1)
effective compatibility analysis of Android apps,
reporting many issues undetected by the
state-of-the-art analysis techniques; and (2)
outperforming other tools in terms of scalability.

\end{itemize}

The rest of this paper is organized as follows.
Section~\ref{sec-background} illustrates various examples of
Android compatibility issues.  Section~\ref{sec-approach}
provides an overview of \@approach to effectively detect
compatibility issues.  Section~\ref{sec-eval} describes our 
empirical study, and
Section~\ref{sec-results} reports the results. Finally, the
paper concludes with a discussion of %performance benefits
current limitations, and an outline of the related
research and future work.
 
